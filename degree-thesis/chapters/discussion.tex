\chapter{Discussion}

\label{Discussion} % For referencing the chapter elsewhere, use \ref{Chapter1}

\section{Other Transformer based architectures}

In the Evaluation Part of this thesis exclusively encoder only Transformers were evaluated 
for mobile malware detection.
While the Encoder is the first and most obvious choice, there are other Transformer 
architectures (see section \ref{sec:rel_transformer} that have potential in 
advancing the field. Those other approaches are discussed in this chapter.

\subsection{Decoder only Transformer based approaches}

Decoder only Transformers operate autoregressively, predicting the next token in a sequence, one by one.
When it comes to the task of malware detection different applications for decoder only models come to mind.

First, a decoder only model can serve as an anomaly detector: 
if a sequence of instructions or API calls from a code class has a low probability under the model, 
it may indicate malicious or obfuscated behavior not seen in benign software 
\cite{tbased_anomalydetection,anomalybased_classification}. 
This approach uses the language models learned distribution to flag outliers.

Second, decoder only models can be integrated into multi step analysis frameworks.
Here one idea setup where one Transformer agent generates malicious behavior (or code variants) 
and another agent (e.g., a detector model) evaluates them. 
Such a generator detector pairing can also work as a form of adversarial training \cite{adversarial_training}.
If such a Framework could be designed without a human in the loop, a reinforcement learning setup 
could be designed. As reinforcement setups have proven very effective in the past \cite{deepseek, alphago}, 
this approach might as well show strong results.
However, we did try to set up a multiagent system as part of this thesis that failed to work.
The setup was based on the crewai \footnote{https://www.crewai.com/} Framework with the 8B and 70B Llama models \cite{llama3modelcard} 
as base LLM. The proof of concept for this approach was to have one agent analyze the manifest.xml file
and ask another agent questions about the java classes that were defined as activities, services, 
receivers and providers. The agents were given a prompt template (see appendix figure \ref{fig:prompt_output}), 
a system template (see appendix figure \ref{fig:system_output}) and tools that they can use.
Unfortunately the approach did not work as the agents were not able to consistently use their 
tools or communicate with other agents and in addition they tend to jump to conclusions once 
a problem came up as shown in the appendix figure \ref{fig:agent_output}. The conclusion here seems to be that while such a 
multiagent architecture seems promising in the field of mobile malware detection, the base 
LLMs yet have to improve in order to reliably work in such a setting.

A third Idea would be to use a model that was trained as a transformer decoder but use it 
to encode a sequence. This could be done by using the normally hidden token embeddings
at each step, before generating the actual output.
The hidden states of the decoder at each step then serve as an encoding of the sequence.
The last hidden state of the decoder could be used as a dense representation 
of the input sequence.
While in theory, Transformer encoders (like BERT) are typically better 
suited for encoding because they 
process the entire sequence in a Bidirectional manner,
this is yet to be proven.

\subsection{Encoder-Decoder Transformer based approaches}

% Deobfuscation
Encoder-decoder Transformers process input sequences with an encoder and generate output 
sequences with a decoder. 
In malware detection, this architecture is helpful for tasks like deobfuscation and 
feature construction. 
One notable example is using a Transformer to reverse code obfuscation: 
Dedek and Scherer \cite{tbased_deobfuscation} trained a character level transformer 
that takes obfuscated PowerShell 
script as input and outputs the original deobfuscated script. 
This approach reports a 92\% full deobfuscation success rate, 
demonstrating how an encoder-decoder can recover malware code that was hidden by attackers obfuscation 
techniques. 
Traditional static approaches struggle with heavily obfuscated code where each variant looks unique, 
but an encoder-decoder model could learn the transformation pattern, 
effectively acting as an automated deobfuscator. 
Despite their strengths, encoder-decoder Transformers can be resource 
intensive and require appropriate training data. In static malware detection, obtaining pairs of 
obfuscated and original code for supervised training may be difficult unless obfuscation techniques 
are simulated. This in turn might limit the deobfuscator to learn only know techniques making it prone to
decreased performance due to concept drift. 

% feature engineering and representation learning
Additionally, enoder-decoder transformers can be used for feature engineering and representation learning.
A big part of mobile malware detection is to properly represent an APK by features that behold 
the relevant information for this task (see chapter \ref{sec:apkrepresentation}).
For instance, the model can produce a longer, expanded form of code from a compressed or packed input, 
or vice versa, effectively handling various forms of coding and code compression. 
This gives encoder-decoder Transformers a degree of flexibility with sequence lengths: 
the encoder creates a compact representation of a long input sequence, and the decoder can iteratively 
construct or analyze it piece by piece, mitigating some challenges of long inputs.
Long code can be effectively segmented or summarized using transformer autoencoders \cite{tbased_codesum}, 
and low level code can be reconstructed in a higher level and 
vice versa \cite{tbased_codetran}.

% Classification
Some authors even believe that transformer autoencoders can be used for the classification task directly
by evaluating the reconstruction error \cite{transformer_malware_overview}. 
The authors argued that it would make them effective in identifying novel or zero-day malware, 
as they can recognize atypical patterns without requiring huge labeled datasets. 
Nonetheless, they may have difficulties with advanced malware that closely resembles benign activity.

\subsection{Graph Transformer based approaches}


Another approach that can leverage either of the different Transformer architectures is based 
on the idea that APKs can be represented as graphs rather than by language.
Examples for these Graphs are control flow graphs of functions \cite{scgformer}, 
call graphs between functions \cite{graphbert}, 
or abstract syntax trees \cite{tbased_codesum}). 
Graph based Transformers apply the attention mechanisms to graph nodes and edges rather than 
linear sequences of tokens. 
For example, Moon et al. \cite{controlflowbert} proposed a Directional Graph Transformer 
to embed control flow graphs for malware classification. 
In this model, each node in a functions control flow graph 
is treated similarly to a token, and attention is computed along the graphs adjacency structure, 
capturing relationships like loops. 
Bu and Cho 
\cite{trplet_graph_tran} 
took this further with a triplet trained graph Transformer, 
using a few shot learning approach with contrastive triplet loss to distinguish 
malware families via their graph embeddings. 
Additionally, the previous in chapter \ref{sec:amd} mentioned approach of the 
"Heterogeneous Temporal Graph Transformer" is a prime example how a graph transformer is 
used at a large company as Tencent \footnote{https://www.tencent.com/en-us/} in production to detect new malware \cite{htgt}.
The graph Transformer processes structural information that sequence models might miss.
These models first convert code to graphs,
then apply a transformer model on the graph. 
The multi head attention is computed over graph neighborhoods, 
showing how strongly one code block is related to another in the context of 
malware behavior 
\cite{transformer_malware_overview}. 
This has shown promise in evaluating relationships within code and to detect which functions 
are critical to malicious behavior. 

\section{Interpretation}

As part of this Thesis, two main experiments were evaluated.
The first experiment (shown in figure \ref{fig:permission_transformer_schema}) represented
APKs by their meta information stored in the manifest.xml file, embedded this representation
in one iteration, and then evaluated the embedding vector of the CLS token via a fully connected layer.
The experiment demonstrated the general effectiveness of the architecture
(shown in figure \ref{fig:bertbased_malwaredetection_schema}) by showing that when processing only
the permissions of an APK, it obtained much better results—up to 76\%
\ref{tab:apk_representation_results_unfrozen}, compared to only 39\% when training a decision tree
on the same permissions \ref{tab:decisiontreepermissions}. 
This drastic increase could be explained by the fact that transformer 
based architectures can interpret custom permissions more effectively than a decision tree due 
to their capability to recognize semantic and contextual relationships, 
rather than just feature counting or simple rule-based splits. 
Another contributing factor might be that transformer based models are inherently better at 
learning complex relationships of permissions, 
as opposed to a decision tree that was constrained by its depth of 15, limiting its complexity.

Another result of the first experiment is that when using the Activities, Permissions and Services 
as APK representation on the time based split Transcend subset, close to 85\% 
F1 score could be achieved as seen in table \ref{tab:full_transcendence_permissions_a_p_s}.
This is a very good result as for example DetectBERT only achieves up to 79\% on the same time based 
split subset \ref{tab:detectbert-results}.
It is important to note that both algorithms were only conducted on a subset of Transcend,
as table \ref{tab:full_transcendence_permissions_a_p} shows that when the Transformer based
architecture is trained on the full transcending dataset, the performance increases from 85\% 
to 91\%, so it is likely that the performance of DetectBERT would increase its performance
as well if it was trained on the full dataset. However, it is unlikely that it would achieve the
F1 score of 97\% that was reported in the paper from the original authors \cite{detectbert}
(at least for the Transcend dataset, for the DexRay set the score might hold.).

One remark that has to be made is, that when training the DetectBERT approach 
(sketched in figure \ref{fig:detectbert_schema}) on the Transcending subset, just as the original work, 
only the DetectBERT decision head was trained and not the DexBERT embedding model.
When comparing the tables \ref{tab:apk_representation_results} and 
\ref{tab:apk_representation_results_unfrozen} it becomes clear that when training the embedding 
model as well as the decision head on the dataset, the performance increases drastically (by
up to 11\% F1 score on the Transcending subset).
Since DetectBERT outperforms the simple approach of the first experiment when the weights 
of its embedding model are frozen by 5\% F1 score (see table \ref{tab:apk_representation_results}) the 
potential of the DetectBERT approach becomes apparent if the encoder model would be trained as well.

The second Experiment that was conducted as part of this thesis was a bit more complex than the first one.
It resembled the DetectBERT approach more in a sense where the APK was represented by code.
Other than smali code as for DetectBERT, the classes.dex file was compiled into java code.
Our approach next embedded the javaclasses one by one, just as the detectBERT approach embedded smali classes.
A distinct difference between the approaches was the handling of classes that exceeded the 
context window size of the embedding model.
DexBERT cuts of the code of the smali class once the token limit is reached.
Our approach uses a sliding window approach where the mean of all the embeddings of one class is 
calculated to generate a class representation.
Both our approach (sketched in figure \ref{fig:java_transformer_schema}) and the 
native DexBERT/DetectBERT approach (sketched in figure \ref{fig:detectbert_schema}) were then evaluated on the same
Transcending subset, both with the DetectBERT decision head.
Our results (table \ref{tab:decision_head_comparison}) with an F1 score of 82\% compared to the 
results we achieved on the DexRay encoder (table \ref{tab:detectbert-results}) with 79\% show the 
potential of our modernBERT based encoding of Java classes, compared to the DexBERT based 
encoding of smali classes.
However, just as the DexBERT approach, our approach has wasted potential as only the decision 
head is trained. Enabling backpropagation on the encoder part is difficult if the APK gets 
encoded in multiple iterations, but it also shows a lot of promise.

Additionally when comparing multiple encoder models as encoders for the simple first experiment, 
our results show that even if training of the encoder model is enabled, only some models 
work as Encoder models while others do not exceed a random guessing as shown in table 
\ref{tab:encoder_model_comparison_cls}.
The BigBird (33\%) and Longformer (33\%) models probably compared so badly next 
to modernBERT (85\%) becasue the 
embedding of the CLS token was used and BigBird and Longformer use sparse Attention, possibly
decreasing the effectiveness of the CLS token as a tradeoff for performance.
However, even when training the last fully connected layer on the average of all embedded tokens, 
modernBERT (81\%) greatly outperforms BigBird (62\%) and Longformer (64\%) as shown in table 
\ref{tab:encoder_model_comparison_average}.

These results show that the architecture that is used for the Encoder has a much bigger 
impact on the overall performance then the architecture of the decision head 
(table \ref{tab:decision_head_comparison}).
We even find that the very simple decision head that is based on the average of all the 
java class representing CLS tokens, is more effective than the much more complex and 
Transformer based DetectBERT approach, indicating that it might be prone to overfitting.


\section{Implication}

All our conducted experiments are built upon prior foundational work that enabled our experiments. 
From the datasets \cite{drebin, transcend, dexray} to the general architecture that was 
inspired by DetectBERT \cite{detectbert} over the transformer models 
that were used as building blocks in that architecture \cite{modernbert,bigbird,longformer}. 
While all of this prior work enabled us to conduct our research, it also limits our findings. 
We showed in Chapter \ref{sec:datasets} how the datasets that we used are inherently flawed by 
being unbalanced (Figure \ref{fig:smali_class_boxplots}, Table \ref{tab:treestump}), 
by not having the same rules for what counts as malware (Figure \ref{fig:euphony_drebin_overlap}), 
or by not representing the full data distribution (Figure \ref{fig:androzoo_temporal}). 
We found that especially the DexRay \cite{dexray} dataset is flawed due to an imbalance between classes in 
code volume. 
All these limiting factors of the datasets decrease the reliability of the algorithms 
that are trained on them, including ours, just as all conducted baselines.

Our experiments demonstrate promising results under controlled conditions, 
but there might be notable gaps when considering real world deployment. 
To explicitly test robustness against concept drift, 
we consistently split our datasets into sequential train and test subsets, 
ensuring our evaluation reflects potential drift over time \cite{tesseract}. 
However, significant concept drift can also occur between different datasets or between 
datasets and the real world, something our sequential splits cannot fully simulate. 
Addressing this would require multiple sufficiently realistic datasets collected from various 
contexts. 
Furthermore, periodic retraining or online learning mechanisms should be integrated 
into practical systems. 

The computational resources demanded by Transformer models could be a problem 
in resource constrained environments such as mobile devices. 
There has been research into compressing and distilling Transformers 
(e.g., MobileBERT \cite{mobilebert} and TinyBERT \cite{tinybert}) 
which could be applied so that a malware detector runs on devices with minimal battery impact. 

Another limiting factor for the experiments of this thesis was of computational nature.
Most experiments were conducted on only a subset of the transcend dataset, 
encompassing approximately 15\% of its total volume. 
This reduction limits the diversity of malware samples encountered during training, 
affecting our models ability to properly distinguish between different kinds of goodware and malware. 
We showed in table \ref{tab:full_transcendence_permissions_a_p} how the
performance of our algorithm increased significantly when training it on the full dataset.

We also showed how new foundational models like modernBERT unlock very simple and yet 
effective architectures. 
With this, We expect the performance of Transformer based algorithms to increase as 
new models with new capabilities emerge. 
We showed how LLama3 \cite{llama3modelcard} is not yet able to be used as an agent 
in a multiagent framework setting, this is another example where more foundational work is needed.

Once sufficient base models are trained, we proved that transfer learning on the specific domain 
is a crucial step in improving their performance.
We expect this to go both ways, once an algorithm is found that reliably classifies APKs that are 
represented by code, this algorithm can next be applied (and transfer learned) in domains
where the task is similar (like code classification) without needing as much data.
With such stepwise specification of models that range from low level base models to
highly specified niche models, many complex problems can be solved.
This is one of the overall strengths of Transformer models since they often base on language (which
is something we produced lots of data over the last few decades), very strong foundational models
can be built. This, in combination with the fact that a lot of problems can be specified in language,
leads to the enormous potential that Transformer models have, in the domain of mobile malware detection
and beyond.

As we also both showed and discussed how Transformers can be used at various steps
of malware detection, this underlines the idea of having multiple highly specialized models used
even for one complex task.
However, while transformers can be used for various subtasks, our experiments regarding the decision
head of our Java Code embeddings (Table \ref{tab:decision_head_comparison}), 
showed how sometimes simpler approaches are both more efficient and effective. 
Here computing the mean of all CLS embeddings was superior to contextualizing them
via attention with DetectBERT.

If Transformers get applied more in real world scenarios as is already the case for
Tencent \cite{htgt}, we also need to make sure that attackers do not adapt to the weaknesses of
Transformers.
One example is that we need to defend the production systems from adversarial examples that have
been proven to be effective against text based Transformer models \cite{adversarial_transformer}.

