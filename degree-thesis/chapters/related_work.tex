\chapter{Related Work} % Main chapter title

\label{related_work} % For referencing the chapter elsewhere, use \ref{Chapter2}

@TODO:  Summary:

\section{Malware Detection}

\subsection{Static, Dynamic, and Network Analysis}
Malware detection approaches include static, dynamic, and network-based methods \cite{vorlesung}.
While this thesis focuses on static analysis, it is worth briefly noting the other approaches for context.

Static analysis examines APK features such as source code and binaries without executing the app.
This makes it highly scalable and suitable for large-scale evaluations.
Prior research like \cite{drebin} and \cite{detectbert} demonstrate the effectiveness and efficiency of static approaches.
However, static analysis does face challenges, including evasion techniques like code obfuscation and polymorphism.

Dynamic analysis, in contrast, observes malware behavior at runtime.
Although it provides richer contextual insights, it requires significant resources and infrastructure and is vulnerable to other evasive strategies, such as sandbox detection.
Similarly, network analysis examines communication patterns and data flows to identify malicious activity.

Hybrid techniques that integrate these methods aim to combine their strengths while mitigating individual limitations.
These, too, face challenges such as computational complexity and integration difficulties.
This thesis, therefore, focuses solely on static analysis as a scalable and efficient approach for malware detection.

\subsection{Evasion Techniques}
Malware authors often employ evasion techniques to bypass detection mechanisms \cite{vorlesung}.
In static analysis, common strategies include manipulating code structures through obfuscation \cite{obfuscation} and polymorphism \cite{polymorhism}.
Obfuscation involves transforming code into a more complex or less readable form to conceal its true purpose, whereas polymorphism allows malware to dynamically change its appearance while maintaining its core functionality.
These approaches exploit the limitations of analyzing code without running it.
While dynamic and network based methods address some of these weaknesses, they introduce their own vulnerabilities, such as susceptibility to sandbox detection and reliance on communication pattern anomalies.
Transformers ability to handle obfuscation and learn robust representations of static code helps mitigate some of these challenges \cite{deobfuscation}.

\subsection{Concept Drift}
Concept drift refers to the gradual evolution of malware patterns, which often leads to a decline in the performance of detection models over time.
Models like Transcend \cite{transcend} have tackled this issue by introducing mechanisms to detect and adapt to concept drift during deployment.
However, these methods are resource-intensive and may struggle to generalize across diverse datasets.
Addressing concept drift remains a critical challenge for static Android malware detection and is one criteria used in this thesis.

\subsection{Android Malware Detection}
Detecting malware on the Android platform presents unique challenges and opportunities due to its open ecosystem and the high volume of apps and malware samples.
The availability of many labeled datasets such as Androzoo, Drebin, Transcend???, and DexRay has driven research in this domain.

Drebin \cite{drebin} represents a leap in Android malware detection, being both a dataset and an approach.
As an SVM-based method, Drebin achieves remarkable results, significantly outperforming other approaches by detecting 94\% of malware samples at a false-positive rate of just 1%.
This demonstrated the high effectiveness of machine learning-based static Android malware detection, striking a balance between accuracy and efficiency.
The Drebin dataset itself provides labeled malware samples with rich feature representations extracted from APK files, such as permissions, intents, and network addresses, making it a foundational resource in the field.

Transcend \cite{transcend}, on the other hand, introduced the critical concept of drift into Android malware detection.
It highlights how models trained on older datasets like Drebin experience performance decay when applied to newer, evolving datasets.
This evolution in malware characteristics, known as concept drift, underscores the need for adaptive retraining mechanisms.
Transcend serves as a benchmark for evaluating the robustness of detection models over time and emphasizes the importance of building systems that can sustain detection accuracy in the face of changing malware landscapes.

Androzoo \cite{androzoo} is a comprehensive repository containing millions of APKs sourced from various app stores and spanning over a decade of Android app development.
The dataset provides significant scale and diversity, enabling researchers to analyze trends in malware evolution and assess the effectiveness of detection methods over time.
In addition to the APKs there was work done on mining metadata for the APKs, enabling further research on the landscape of Android Apps.

DetectBERT \cite{detectbert} is a novel transformer based approach to Android malware detection that builds upon DexBERT \cite{dexbert}, another transformer based model designed for class level analysis of Android Smali code.
Using a Correlated Multiple Instance Learning (c-MIL) framework, DetectBERT aggregates Smali class embeddings of DexBERT into an app level representation.
Evaluated on the DexRay \cite{dexray} dataset, which comprises 158,803 apps, DetectBERT reported competitive performance metrics, including 97\% accuracy and an F1 score of 97\% while also demonstrated robustness against concept drift, maintaining high detection rates when tested on newer malware samples.
Additionally, DetectBERT reports to be computationally efficient, requiring only 2GB of GPU memory and achieving inference times of 0.005 seconds per app.

